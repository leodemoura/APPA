\documentclass{llncs}

\usepackage{url,amsmath,amssymb,a4wide}

% Dissemination Plans and Important Dates:

% - Your tutorial shall be acompanied by a paper, 
% which will be published as a chapter of a book 
% in the "Logic and Foundations of Mathematics" series 
% by College Publications. 

% - Please send us a good, though not necessarily final, 
% version of your paper ** BEFORE 1st OF JUNE **. 
% This will give us (and College Publications) sufficient time 
% to produce printed copies of a preliminary version of the book, 
% to be distributed to registered participants.

% - Afterwards there will be plenty of time to improve
% the paper/chapter and incorporate feedback gained during the event. 


% General remarks:

% - The target audience (for both the tutorials and 
% the accompanying papers) consists of Ph.D. students, 
% young post-docs and researchers from other logic-related communities.

% - Avoid obscurity and clarify concepts that might be 
% unknown to people from other communities.

% - Strive for a self-contained paper, but be concise and 
% cite papers where readers may find more information.

% - Do not hesitate to build bridges between your domain and 
% other proof-related communities if you have some ideas to do so. 
% These bridges should lead to interesting discussions during the workshop.

% - You are encouraged to compare what is done in your community 
% with what is done in other communities. 
% Try to understand and explain why your community does things differently.
% This could lead to interesting discussions too.

% - This template aims at ensuring a 
% reasonably uniform style for all speakers. 
% Nevertheless, feel free to deviate from the template if you need. 
% We are aware that not all questions are applicable to all speakers.


% We hope the questions here will guide you 
% in the production of your tutorial.

% If anything is unclear, if you have any questions, contact us!

% Thank you!


\title{ ToDo } 
% Please select a general title that reflects 
% the community you are going to represent in the event.


\author{
  ToDo Todo \inst{1} 
  \and 
  ToDo Tada \inst{2}
}

\authorrunning{T.\~Todo \and T.\~Tada}

\institute{
  ToDo\\
  \email{ }
  \and 
  ToDo \\
  \email{ }
}

\begin{document}

\maketitle


\section{Introduction}

% Introduce your application domain and briefly discuss why proofs are important
% in this area.
% What are (some of) the most impressive achievements in these application
% domains?
% Can you be considered as a pioneer in the introduction of formal proofs in
% this area? If so, which application was the starting point?

% Briefly describe the historical context in which 
% your application domain originated, 
% with a particular focus on the 
% co-evolution of proofs (and proof systems and proof-generating tools) 
% and your application of proofs. What came first? Proofs? Or your application? 
% Has your application contributed to shape the evolution of proof-generating tools? 
% Has the evolution of proof-generating tools enabled anything 
% in particular in your application area?


\section{Deduction Tools and Proof Systems}

% Which *kinds* of proof-generating deduction tools are most commonly used in your application area?
% Which concrete implemented deduction tools are most often used?

% Which proof systems are used by these tools? 
% Is this your preferred proof system? 
% Or would you (your applications) prefer proofs in another proof system?

% In case you have a small, interesting and reasonably human-readable example 
% of proof used in your application domain, could you copy-paste it here?

% Are you satisfied with the proofs generated by the tools you use? 
% Are they detailed enough? Are your proofs intended to be read?
% Is there anything missing? What could be improved? 


% Are there other kinds of deduction tools whose potential in your application
% area remains to be explored? Are there other kinds of tools that are unlikely
% to be useful in your application area? Do you think that it could be fruitful
% for your application domain to combine several deduction tools or one
% deduction tool with completely different systems (e.g. symbolic computation
% tools, model-checkers, etc.)?

\section{Using Proofs}

% How do you use proofs? 

% Are proofs just certificates of correctness? 
% Or do they contain extra information 
% (e.g. unsat cores, interpolants, witnesses, programs) 
% that is relevant for your application domain?

% Which algorithms/techniques do you use to post-process, 
% transform or extract information from proofs?

% Do you need to extract computable contents from your specifications and your
% proofs? If so, how do you proceed? Do you try to sytematically turn your
% specifications into executable ones or do you use specific extraction
% mechanisms (from functional or inductive specifications)?


% Are your proofs intended to be attached to your code (like in proof-carrying
% code)? If so, what kind of software is used to verify the formal proofs that
% accompany the code of your applications?

\section{Comparison of Proofs}

% In your application domain, when are two proofs considered to be the same? 
% Is this question relevant?

% And when is a proof considered better than another proof of the same theorem?

% Do you think that the proofs of your application would have been better
% expressed or more easily completed in another proof system or tool? If a
% similar proof of your application has been developed in another proof tool,
% what are weak and strong points of your development compared to this
% alternative development?

\section{Trends and Open Problems}

% What are the current trends, hot topics and open problems in the further
% development of your application domain? Focus on the use of proofs.

% Could you outline your vision for the next ten years of the place of formal
% proofs in you application domain?

\section{Conclusions}




\bibliographystyle{plain}
\bibliography{Bibliography}


\end{document}