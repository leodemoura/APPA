\documentclass{llncs}

\usepackage{url,amsmath,amssymb,a4wide}

% Dissemination Plans and Important Dates:

% - Your tutorial shall be acompanied by a paper, 
% which will be published as a chapter of a book 
% in the "Logic and Foundations of Mathematics" series 
% by College Publications. 

% - Please send us a good, though not necessarily final, 
% version of your paper ** BEFORE 1st OF JUNE **. 
% This will give us (and College Publications) sufficient time 
% to produce printed copies of a preliminary version of the book, 
% to be distributed to registered participants.

% - Afterwards there will be plenty of time to improve
% the paper/chapter and incorporate feedback gained during the event. 


% General remarks:

% - The target audience (for both the tutorials and 
% the accompanying papers) consists of Ph.D. students, 
% young post-docs and researchers from other logic-related communities.

% - Avoid obscurity and clarify concepts that might be 
% unknown to people from other communities.

% - Strive for a self-contained paper, but be concise and 
% cite papers where readers may find more information.

% - Do not hesitate to build bridges between your domain and 
% other proof-related communities if you have some ideas to do so. 
% These bridges should lead to interesting discussions during the workshop.

% - You are encouraged to compare what is done in your community 
% with what is done in other communities. 
% Try to understand and explain why your community does things differently.
% This could lead to interesting discussions too.

% - This template aims at ensuring a 
% reasonably uniform style for all speakers. 
% Nevertheless, feel free to deviate from the template if you need. 
% We are aware that not all questions are applicable to all speakers.


% We hope the questions here will guide you 
% in the production of your tutorial.

% If anything is unclear, if you have any questions, contact us!

% Thank you!


\title{ ToDo } 
% Please select a general title that reflects 
% the community you are going to represent in the event.


\author{
  ToDo Todo \inst{1} 
  \and 
  ToDo Tada \inst{2}
}

\authorrunning{T.\~Todo \and T.\~Tada}

\institute{
  ToDo\\
  \email{ }
  \and 
  ToDo \\
  \email{ }
}

\begin{document}

\maketitle


\section{Introduction}

% Introduce your kind of proof system, proof-theoretical framework 
% or technique for improving proof systems/frameworks. 

% Briefly mention the technical problems (both theoretical and practical) 
% of previous approaches that led you and your community 
% to pursue something different or better.

% Briefly describe the historical context in which 
% your approach was developed.


\section{Proof System(s)}

% Please show the inference rules of (a variant of) 
% your kind of proof system explicitly. 
% Use, for example, proof.sty or bussproofs.sty.  

% In case you advocate a general approach that is applicable to many 
% kinds of proof systems, 
% please choose a few concrete proof systems to illustrate the approach.
% Is there any kind of proof system to which your approach is 
% not applicable?

% Please show small but representative example(s) of concrete proof(s) 
% using the proof system(s) defined above. The example should show the
% peculiarities of the proof system(s).



\section{Proof-Theoretical Properties}

% Use this section to define and discuss proof-theoretical properties 
% that you consider desirable in a proof system. 
% Explain why you and your community consider them desirable.

% What would be the features/properties of an ideal proof system? 
% Which features/properties are not so important?

% Focus on properties that are enjoyed by your 
% system/approach/technique but not by others.

% According to which criteria is your 
% proof system/technique/approach better than others? 
% According to which criteria (if any) could it be considered worse?

% Are there any results showing that your proof system can simulate other systems?


\section{Pragmatic Properties}

% Please discuss and compare how your 
% proof-system/technique/approach behaves 
% with respect to various practical concerns/criteria:

% Is it adequate/beneficial for proof search? 
% Would automated provers based on it be faster? less memory-consuming?

% Is it adequate/beneficial for representing proofs 
% after they have been found?

% Does it make proof-checking easier? or harder?

% Does it make proofs more human-readable?


\section{Comparison of Proofs}

% In your proof system, when should two proofs be considered the same? 

% When should a proof be considered better than another proof of the same theorem?


\section{Tools}

% For what kind of deduction tools (e.g. sat-solvers, smt-solvers, 
% FOL automated theorem provers, HOL automated theorem provers, 
% proof assistants,...) are the proof 
% systems/techniques/approaches discussed above most appropriate?

% Are there implementations of deduction tools based on your proof system? 

% If yes, please mention/cite them. 
% Briefly discuss how well they implement the proof-theoretical 
% principles/techniques/systems discussed before. 
% Are there any discrepancies worth mentioning?

% If there are no deduction tools based on your proof systems/frameworks/techniques, why not (yet)?
% Are there technical difficulties that make 
% your proof system/technique/approach harder to implement? 
% Are there sociological/historical barriers to its adoption?



\section{Proof Applications}

% Which application domains have used your proof system, framework or technique? Have there been any ground-breaking achievements in these application domains? How are the proofs used? 


\section{Trends and Open Problems}

% What are the current trends, hot topics and open problems in the further development of your proof systems, frameworks or approaches?


\section{Conclusions}




\bibliographystyle{plain}
\bibliography{Bibliography}


\end{document}